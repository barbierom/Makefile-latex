\documentclass[8pt]{beamer}

\setbeamertemplate{navigation symbols}{}

\usepackage[sc]{mathpazo} % serif and math mode
\usepackage{tgadventor}		% sans serif font
\renewcommand*\familydefault{\sfdefault} % Base font of the document is sans serif
\DeclareMathSizes{9}{10.3}{6.5}{6.5} % Match math and ss text font size
\DeclareMathSizes{8}{9.5}{6}{6}

\usepackage[T1]{fontenc}       
\usepackage[utf8]{inputenc}    % pour les accents (mettre latin1 pour windows au lieu de utf8)
%\usepackage[frenchb]{babel}    % le documents est en français

\usepackage{natbib}
%\newcommand{\newblock}{}  % Ce truc met une erreur - Je ne sais plus à quoi ça sert...

\usepackage{amsmath}           % un packages mathématiques
\usepackage{latexsym}                % to get LASY symbols     
\usepackage{amsfonts}  
\usepackage{amssymb}
\usepackage{amsthm}
\usepackage{dcolumn}
\usepackage{color}
\usepackage{mathrsfs}
\usepackage{xcolor}            % pour définir plus de couleurs et colorer les math
\usepackage{graphicx} 
\usepackage{tikz}  
%%%%%
% For every picture that defines or uses external nodes, you'll have to
% apply the 'remember picture' style. To avoid some typing, we'll apply
% the style to all pictures.
\tikzstyle{every picture}+=[remember picture]

% By default all math in TikZ nodes are set in inline mode. Change this to
% displaystyle so that we don't get small fractions.
\everymath{\displaystyle}
%%%%%

\usepackage{xmpmulti}
\usepackage{pgf}
\usepackage[percent]{overpic}		% pour inclurer text et.ou figure par dessus une figure remplacer "abs" par "percent" pour relatif

\usepackage{beamerthemesplit}

\usepackage[absolute,showboxes,overlay]{textpos}     % déclaration du package
\textblockorigin{0cm}{0cm}                               % origine des positions
%\TPshowboxestrue                                     % affiche le contour des textblock
\TPshowboxesfalse                                    % n'affiche pas le contour des textblock\\
 
\setbeamercovered{transparent}
\mode<presentation>
\usetheme[numbers,totalnumber,compress,sidebarshades]{Madrid}
\setbeamertemplate{footline}[frame number]
 
%\definecolor{bellblue}{RGB}{51,102,166}
\definecolor{bellblue}{RGB}{27,136,202}

\usecolortheme[named=bellblue]{structure}
\useinnertheme{circles}
\usefonttheme[onlymath]{serif}
\setbeamercovered{transparent}

%\setbeamercolor{math text}{fg=green!45!black}
\setbeamercolor{title}{bg=bellblue}
\setbeamercolor{frametitle}{bg=bellblue}
%\setbeamercolor{footline}{bg=black,fg=white}
%\setbeamercolor{author in head/foot}{bg=black,fg=white}
%\setbeamercolor{title in head/foot}{bg=white!45!black}
\setbeamercolor{block title}{bg=white!45!black}
%\setbeamercolor{block title example}{bg=white!45!black}
%\setbeamercolor{block title alerted}{bg=white!15!black}

\setbeamercolor{block title}{fg = bellblue, bg = white}
\setbeamercolor{block body}{fg = white, bg = bellblue}
\setbeamertemplate{blocks}[rounded][shadow]

%\setbeamerfont{frametitle}{series=\bfseries,size=\normalsize}
\setbeamerfont{frametitle}{size=\Large}

%the needed packages 
\usepackage[absolute]{textpos} 
%\usepackage[colorgrid,texcoord]{eso-pic}
%adjust the TPHorizModule and TPHorizModule units to the displayed mm %grid 
\TPGrid{128}{96} 
%puts a graphic at the absolute position described by the grid 
%#1 x, #2 y, #3 width, #4 height, #5 graphic 
\newcommand\putpic[5]{% 
\begin{textblock}{#3}(#1,#2) 
\includegraphics[width=#3\TPHorizModule, 
height=#4\TPVertModule]{#5} 
\end{textblock} 
} 

%\renewcommand{\phi}{\varphi}
\renewcommand{\>}{\right \rangle}
\newcommand{\<}{\left \langle}
\newcommand{\ket}[1]{\left |#1\>}
\newcommand{\bra}[1]{\<#1\right |}
\newcommand{\be}{\begin{equation}}
\newcommand{\ee}{\end{equation}}
\newcommand{\bea}{\begin{eqnarray}}
\newcommand{\eea}{\end{eqnarray}}
\newcommand{\Real}{\mathbb{R}}

\newcommand{\btVFill}{\vskip0pt plus 1filll} % to push note at the bottom of the frame

\newcommand{\backupbegin}{
   \newcounter{framenumberappendix}
   \setcounter{framenumberappendix}{\value{framenumber}}
}
\newcommand{\backupend}{
   \addtocounter{framenumberappendix}{-\value{framenumber}}
   \addtocounter{framenumber}{\value{framenumberappendix}} 
}

\newcommand{\bellRef}[3]{\textcolor{gray}{\textit{{\footnotesize #1, #2 (#3)}}}}

\newcommand*\circled[3]{\tikz[baseline=(char.base)]{
            \node[shape=circle,draw=#1,opacity=#2,inner sep=2pt] (char) {\textcolor{#1}{\textbf{#3}}};}}

\newcommand{\bellhl}[1]{\textcolor{bellblue}{\textbf{#1}}}


